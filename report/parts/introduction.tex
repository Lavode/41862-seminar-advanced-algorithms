\chapter{Introduction}
\label{chapter:introduction}

This report aims to provide the reader an introduction into the fast integer
multiplication algorithm proposed by Schönhage and Strassen in 1971.
\autocite{schonhageSchnelleMultiplikationGrosser1971}

% TODO make sure this is actually correct :)
It will start with a discussion of the problem to be solved and historical
solutions in chapter \ref{chapter:introduction}. Chapter
\ref{chapter:mathematical_background} will introduce the required mathematical
background. Chapter \ref{chapter:fast_integer_multiplication} will then introduce
the algorithm, and explain its workings. Finally we will conclude in chapter
\ref{chapter:conclusion}.

\section{Problem statement}

The problem the Schönhage-Strassen algorithm intends to solve is to, given two
integers $x, y \in \mathbb{Z}$, calculate their product $z \coloneqq x \cdot
y$.

\section{Historical algorithms}

\subsection{Long multiplication}

One of the most common algorithm for multiplication taught in school is known
as `long multiplication'. Given a representation of the multiplicand and
multiplier in some base --- commonly base 10 --- each digit of the multiplicand
is multiplied by each digit of the multiplier. All intermediary products
encoding for the same digit of the chosen base will then be summed up. Finally
a potential carry propagation takes place, yielding a representation of the
product in the chosen base.

It is trivial to see that this algorithm requires $O(n^2)$ multiplication and additions.

% TODO Karatsuba / Toom-Cook?
