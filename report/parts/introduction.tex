\chapter{Introduction}
\label{chapter:introduction}

This report aims to provide the reader an introduction into the fast integer
multiplication algorithm proposed by Schönhage and Strassen in 1971.
\autocite{schonhageSchnelleMultiplikationGrosser1971}

It will start with a discussion of the problem to be solved and historical
solutions in chapter \ref{chapter:introduction}. Chapter
\ref{chapter:mathematical_background} will introduce the required mathematical
background, with a focus on convolution theory and the Discrete Fourier
Transform.. Chapter \ref{chapter:integer_multiplication_convolution} will show
how to model the problem of integer multiplication as a convolution problem,
along with a basic algorithm to solve it. Chapter
\ref{chapter:schoenhage_strassen} will then introduce the Schönhage-Strassen
algorithm, and explain various of its details. Finally we will conclude in
chapter \ref{chapter:conclusion}.

\section{Problem statement}

The problem the Schönhage-Strassen algorithm intends to solve is to, given two
integers $x, y \in \mathbb{Z}$, calculate their product $z \coloneqq x \cdot
y$.

\section{Historical development}

\subsection{Long multiplication}

One of the most common algorithms for multiplication taught in school is known
as `long multiplication'. Given a representation of the multiplicand and
multiplier in some base --- commonly base 10 --- each digit of the multiplicand
is multiplied by each digit of the multiplier. All intermediary products
encoding for the same digit of the chosen base will then be summed up. Finally
a potential carry propagation takes place, yielding a representation of the
product in the chosen base.

It is trivial to see that this algorithm requires $O(n^2)$ multiplication and
additions.

\subsection{Work leading up to Schönhage-Strassen}

For a while it had been conjectured that an asymptotic complexity $\Omega(n^2)$
was ideal for the problem of integer multiplication. In 1962, Karatsuba found
an algorithm where one multiplication of two $2n$-digit numbers could be
replaced with three multiplications of $n$-digit numbers. This lead to an
asymptotic complexity of $\approx O(n^{1.58})$. In 1963 and 1966, Toom and Cook
generalized Karatsuba's approach to a complexity of $O(n^{1.46})$. In 1971
finally, Schönhage and Strassen proposed an algorithm with complexity $O(n
\log(n) \log(\log(n)))$ which we will discuss
here.
