\chapter{Conclusion}
\label{chapter:conclusion}

This report has provided an overview of the Schönhage-Strassen algorithm. It
has provided an introduction into convolution theory, and has shown how to
calculate convolutions using the DWT and DFT.

It has also shown how to model normal and modular integer multiplication as a
convolution problem, which can be efficiently calculated by means of the DFT
and DWT.

Lastly it has shown the workings of the Schönhage-Strassen multiplication
algorithm, which works in $\setint_{2^n + 1}$ the ring of integers modulo $2^n
+ 1$. It has shown how this circumvents rounding problems, allows using fast
integer arithmetic, and allows speeding up multiplications by roots of unity
as they can be chosen to be powers of two.

Lastly it should be noted that the Schönhage-Strassen algorithm has a
relatively large overhead. As such its asymptotic complexity does not start to
be advantageous, until the integers to be multiplied are rather large. As a
rule of thumb, the following multiplication algorithms are best used for
integers of various sizes:

\begin{description}
		\item[Up to hardware word size (e.g. $2^{64}$)] Hardware multiplication
		\item[After 64 bits] Long multiplication
		\item[After 512 bits] Karatsuba / Toom-Cook
		\item[After $2^{16}$ bits] Schönhage-Strassen multiplication
		\item[After $2^{64}$ bits] Fürer's algorithm
\end{description}

As a final note: In 2019, Harvey and van der Hoeven published an algorithm for
integer multiplication with an asymptotic complexity of $O(n \log(n))$.
\autocite{harveyIntegerMultiplicationTime2021} While their algorithm is not
practical for any computations fitting into the observable universe, they did
thereby reach the complexity which Schönhage and Strassen had conjectured to be
optimal in 1971.
